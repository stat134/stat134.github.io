\documentclass[11pt]{amsart}

\usepackage{amsmath}
\usepackage{amsthm}
\usepackage{amsfonts}
\usepackage{amssymb}
\usepackage{enumerate}

\usepackage{graphics}
\usepackage{epsfig}
\usepackage{color}
\usepackage{verbatim}
%\parindent = 5 pt
%\parskip = 12 pt
%\oddsidemargin 0.1in \evensidemargin 0.1in \textwidth=7in
%\textheight=8.5in \itemsep=0in
%\parsep=0.1in

\textwidth 16cm \textheight 24cm \oddsidemargin 0.1cm
\evensidemargin 0.1cm \topmargin -0.8cm
%\def\yy{{\hbox{\bf y}}}
%\newcommand\eps{\varepsilon}
%\newcommand\tr{\operatorname{tr}}
\newcommand\diag{\operatorname{diag}}\newcommand\hess{\operatorname{Hess}}
\newcommand\divergence{\operatorname{div}}
\newcommand\ad{\operatorname{ad}}
%\newcommand\dist{\operatorname{dist}}
%\renewcommand\span{\operatorname{span}}
\newcommand\rank{\operatorname{rank}}
%\renewcommand\Re{\operatorname{Re}}
%\renewcommand\Im{\operatorname{Im}}
\newcommand\g{{\mathbb{g}}}
\newcommand\R{{\mathbb{R}}}
\newcommand\C{{\mathbb{C}}}
\newcommand\Z{{\mathbf{Z}}}
\newcommand\D{{\mathbf{D}}}
\newcommand\Q{{\mathbf{Q}}}
\newcommand\G{{\mathbf{G}}}
\newcommand\I{{\mathbf{I}}}
\renewcommand\P{{\mathbf{P}}}
\newcommand\E{{\mathbf{E}}}
\newcommand\U{{\mathbf{U}}}
\newcommand\A{{\mathbf{A}}}
\newcommand\Var{\mathbf{Var}}
\renewcommand\Im{{\operatorname{Im}}}
\renewcommand\Re{{\operatorname{Re}}}
\newcommand\eps{{\varepsilon}}
\newcommand\trace{\operatorname{trace}}
\newcommand\supp{\operatorname{supp}}
\newcommand\tr{\operatorname{tr}}
\newcommand\dist{\operatorname{dist}}
\newcommand\Span{\operatorname{Span}}
\newcommand\sgn{\operatorname{sgn}}
\newcommand\dive{\operatorname{div}}
\renewcommand\a{x}
\renewcommand\b{y}

% \newcommand\bm{{\mathbf{m}}}
\newcommand\dd{\partial}
\newcommand\smd{\hbox{smd}}
\newcommand\Ba{{\mathbf a}}
\newcommand\Bb{{\mathbf b}}
\newcommand\Bc{{\mathbf c}}
\newcommand\Bd{{\mathbf d}}
\newcommand\Be{{\mathbf e}}
\newcommand\Bf{{\mathbf f}}
\newcommand\Bg{{\mathbf g}}
\newcommand\Bh{{\mathbf h}}
\newcommand\Bi{{\mathbf i}}
\newcommand\Bj{{\mathbf j}}
\newcommand\Bk{{\mathbf k}}
\newcommand\Bl{{\mathbf l}}
\newcommand\Bm{{\mathbf m}}
\newcommand\Bn{{\mathbf n}}
\newcommand\Bo{{\mathbf o}}
\newcommand\Bp{{\mathbf p}}
\newcommand\Bq{{\mathbf q}}
\newcommand\Bs{{\mathbf s}}
\newcommand\Bt{{\mathbf t}}
\newcommand\Bu{{\mathbf u}}
\newcommand\Bv{{\mathbf v}}
\newcommand\Bw{{\mathbf w}}
\newcommand\Bx{{\mathbf x}}
\newcommand\By{{\mathbf y}}
\newcommand\Bz{{\mathbf z}}
\newcommand\p{{\mathbf p}}
\newcommand\bi{{\mathbf i}}
%

\newcommand\BA{{\mathbf A}}
\newcommand\BB{{\mathbf B}}
\newcommand\BC{{\mathbf C}}
\newcommand\BD{{\mathbf D}}
\newcommand\BE{{\mathbf E}}
\newcommand\BF{{\mathbf F}}
\newcommand\BG{{\mathbf G}}
\newcommand\BH{{\mathbf H}}
\newcommand\BI{{\mathbf I}}
\newcommand\BJ{{\mathbf J}}
\newcommand\BK{{\mathbf K}}
\newcommand\BL{{\mathbf L}}
\newcommand\BM{{\mathbf M}}
\newcommand\BN{{\mathbf N}}
\newcommand\BO{{\mathbf O}}
\newcommand\BP{{\mathbf P}}
\newcommand\BQ{{\mathbf Q}}
\newcommand\BS{{\mathbf S}}
\newcommand\BT{{\mathbf T}}
\newcommand\BU{{\mathbf U}}
\newcommand\BV{{\mathbf V}}
\newcommand\BW{{\mathbf W}}
\newcommand\BX{{\mathbf X}}
\newcommand\BY{{\mathbf Y}}
\newcommand\BZ{{\mathbf Z}}
\renewcommand\Pr{{\mathbf P }}

%cal letter

\newcommand\CA{{\mathcal A}}
\newcommand\CB{{\mathcal B}}
\newcommand\CC{{\mathcal C}}
\newcommand\CD{{\mathcal D}}
\newcommand\CE{{\mathcal E}}
\newcommand\CF{{\mathcal F}}
\newcommand\CG{{\mathcal G}}
\newcommand\CH{{\mathcal H}}
\newcommand\CI{{\mathcal I}}
\newcommand\CJ{{\mathcal J}}
\newcommand\CK{{\mathcal K}}
\newcommand\CL{{\mathcal L}}
\newcommand\CM{{\mathcal M}}
\newcommand\CN{{\mathcal N}}
\newcommand\CO{{\mathcal O}}
\newcommand\CP{{\mathcal P}}
\newcommand\CQ{{\mathcal Q}}
\newcommand\CS{{\mathcal S}}
\newcommand\CT{{\mathcal T}}
\newcommand\CU{{\mathcal U}}
\newcommand\CV{{\mathcal V}}
\newcommand\CW{{\mathcal W}}
\newcommand\CX{{\mathcal X}}
\newcommand\CY{{\mathcal Y}}
\newcommand\CZ{{\mathcal Z}}

%number theory

\newcommand\condo{{\bf C0}}
\newcommand\condone{{\bf C1}}

\newcommand\N{{\mathbb N}}
\newcommand\BBQ {{\mathbb Q}}
\newcommand\BBI {{\mathbb I}}
\newcommand\BBR {{\mathbb R}}
\newcommand\BBZ {{\mathbb Z}}
\newcommand\BBC{{\mathbb C}}
% tilde

\newcommand\ta{{\tilde a}}
\newcommand\tb{{\tilde b}}
\newcommand\tc{{\tilde c}}
\newcommand\td{{\tilde d}}
\newcommand\te{{\tilde e}}
\newcommand\tf{{\tilde f}}
\newcommand\tg{{\tilde g}}
% \newcommand\th{{\tilde h}}
\newcommand\ti{{\tilde i}}
\newcommand\tj{{\tilde j}}
\newcommand\tk{{\tilde k}}
\newcommand\tl{{\tilde l}}
\newcommand\tm{{\tilde m}}
\newcommand\tn{{\tilde n}}
%\newcommand\to{{\tilde o}}
\newcommand\tp{{\tilde p}}
\newcommand\tq{{\tilde q}}
\newcommand\ts{{\tilde s}}
%\newcommand\tt{{\tilde t}}
\newcommand\tu{{\tilde u}}
\newcommand\tv{{\tilde v}}
\newcommand\tw{{\tilde w}}
\newcommand\tx{{\tilde x}}
\newcommand\ty{{\tilde y}}
\newcommand\tz{{\tilde z}}
\newcommand\ep{{\epsilon}}
%\newcommand\trace{{\operatorname{trace}}}
\newcommand\sign{{\operatorname{sign}}}
%\newcommand\hess{{\operatorname{Hess}}}
\newcommand\Dyson{{\operatorname{Dyson}}}
\renewcommand\th{{\operatorname{th}}}






\def\x{{\bf X}}
\def\y{{\bf Y}}
%\def\I#1{{\bf 1}_{#1}}
\def\mb{\mbox}

% \swapnumbers
% \pagestyle{headings}
\parindent = 5 pt
\parskip = 12 pt

\theoremstyle{plain}
  \newtheorem{theorem}{Theorem}[section]
  \newtheorem{conjecture}[theorem]{Conjecture}
  \newtheorem{problem}[theorem]{Problem}
  \newtheorem{assumption}[theorem]{Assumption}
  \newtheorem{heuristic}[theorem]{Heuristic}
  \newtheorem{proposition}[theorem]{Proposition}
  \newtheorem{fact}[theorem]{Fact}
  \newtheorem{lemma}[theorem]{Lemma}
  \newtheorem{corollary}[theorem]{Corollary}
  \newtheorem{claim}[theorem]{Claim}
 % \newtheorem{problem} [theorem]{Question}

\theoremstyle{definition}
  \newtheorem{definition}[theorem]{Definition}
  \newtheorem{example}[theorem]{Example}
  \newtheorem{remark}[theorem]{Remark}
  \numberwithin{equation}{section}

\begin{document}
\title{Math 104 - Weekly assignment 3}
\author{Due 16 September 2016, by 16:00}
\maketitle

\begin{enumerate}

\item
\begin{enumerate}[(i)]
\item Let $b_n=\frac{1}{\sqrt{n^2+1}}+\frac{1}{\sqrt{n^2+2}}+\ldots +\frac{1}{\sqrt{n^2+n}}$, for all $n\in\N$. What is the limit of $(b_n)_{n\in\N}$?

(Hint: Be very careful here: each of the terms above converges to 0, but that doesn't immediately imply that their sum will converge to 0: this is because the number of terms *also* depends on $n$. For instance, think of $1=\frac{1}{n}+\ldots +\frac{1}{n}$; $\frac{1}{n}\rightarrow 0$, but $1\nrightarrow 0$. Use that
$$\frac{1}{\sqrt{n^2+1}}\leq \frac{1}{\sqrt{n^2+k}}\leq \frac{1}{\sqrt{n^2+n}}\text{, for all }k=1,\ldots,n.)
$$
\vspace{0.1in}
\item Let $c_n=\frac{1+2^2+3^3+\ldots+n^n}{n^n}$. Show that $c_n\rightarrow 1$.

(Hint: Use the sandwich lemma, but don't allow such a loss as for the previous question: this time exploit that the power in the denominator is much larger than the powers of the terms in the numerator.)
\vspace{0.1in}
\item Let $a_1,\ldots,a_k>0$. Show that $\sqrt[n]{a_1^n+\ldots + a_k^n}\rightarrow \max\{a_1,\ldots,a_k\}$ as $n\rightarrow +\infty$.
\end{enumerate}
\vspace{0.4in}

\item The aim of this exercise is to show (part of) the algebra of limits regarding convergence to $+\infty$:
\begin{enumerate}[(i)]
\item Show that, if $a_n\rightarrow +\infty$, and $a_n\neq 0$ for all $n\in\N$, then $\frac{1}{a_n}\rightarrow 0$.
\item Show that, if $a_n\rightarrow 0$ and $a_n>0$ for all $n\in\N$, then $\frac{1}{a_n}\rightarrow +\infty$. (Note that $a_n>0$ for all $n\in\N$ is required; look for instance at $\frac{(-1)^n}{n}$ for all $n\in\N$.)
\item Show that, if $a_n\rightarrow +\infty$ and $(b_n)_{n\in\N}$ is bounded from below, then $a_n+b_n\rightarrow +\infty$.
\item Show that, if $a_n\rightarrow +\infty$, then $a_n^2\rightarrow +\infty$.
\item Show that, if $a_n\rightarrow +\infty$, and $a_n>0$ for all $n\in\N$, then $a_n^{\frac{1}{k}}\rightarrow +\infty$ as $n\rightarrow +\infty$, for any fixed $k\in\N$.
\item Show that, if $a_n>b_n$ for all $n\in\N$, and $b_n\rightarrow +\infty$, then $a_n\rightarrow +\infty$.
\item Show that, if $a_n\rightarrow +\infty$ and $\lambda >0$, then $\lambda a_n\rightarrow +\infty$.
\end{enumerate}
\vspace{0.1in}

(Hint: These follow easily from the definition of limits, but keep in mind also if you can use any of these to show another of these. For instance, you may want to use (ii) a lot.)
\vspace{0.4in}

\item Provide a proof of the root test. (Start the same way as for the ratio test).
\vspace{0.4in}

\item Are the following true or false? Justify your answers.
\begin{enumerate}[(i)]
\item If $a_n>0$ for all $n\in\N$ and $(a_n)_{n\in\N}$ is not bounded from above, then $a_n\rightarrow +\infty$.
\item $a_n\rightarrow +\infty$ $\Leftrightarrow$ for all $M>0$, there exist infinitely many terms of $(a_n)_{n\in\N}$ larger than $M$. (Justify each direction here.)
\end{enumerate}
\vspace{0.4in}

\item Find the limit of $(a_n)_{n\in\N}$ (as $n\rightarrow +\infty$), when:
\begin{enumerate}[(i)]
\item $a_n=\frac{n^5+4n^3-3}{2n^8+5n}$, for all $n\in\N$.
\vspace{0.1in}
\item $a_n=\frac{n^2+\sqrt{n^{\frac{5}{2}}+1}}{n^4+3}$, for all $n\in\N$.
\vspace{0.1in}
\item $a_n=\frac{2^n+n}{3^n-n}$, for all $n\in\N$.
\vspace{0.1in}
\item $a_n=\sqrt[n]{10n^{91}}$, for all $n\in\N$.
\vspace{0.1in}
\item $a_n=\sqrt[n]{n^7-8n^3+11n^2-6}$, for all $n\in\N$, $n\geq 2$.
\vspace{0.1in}
\item $a_n=\frac{n^n}{n!}$, for all $n\in\N$.
\vspace{0.1in}
\item $a_n=\frac{n^k}{n!}$, for all $n\in\N$, where $k\in\N$ is fixed (imagine it like 2 or 3).
\vspace{0.1in}
\item $a_n=\left(1+\frac{1}{n}\right)^{n^2}$, for all $n\in\N$.
\vspace{0.1in}
\item $a_n=\left(\frac{1+2^n}{n^2}\right)^n$, for all $n\in\N$.
\vspace{0.1in}
\item $a_n=\sqrt{n^2+2}-\sqrt{n^2+1}$, for all $n\in\N$. (Hint: the limit will be 0. You see that this is an indeterminate form (the limit of each of the terms in the difference is $+\infty$, and $+\infty - (+\infty)$ doesn't make any sense. Multiply and divide with an appropriate quantity (depending on $n$), which will eliminate the problem.)
\vspace{0.1in}

(Hint: For each of these, it is up to you to choose the technique you think fits best: extracting the largest power, or the ratio or root tests, or the sandwich lemma, the definition of limit, any of the other questions in the assignment, etc.)
\end{enumerate}
\vspace{0.4in}

\item Let $a_1=\sqrt{2}$ and $a_{n+1}=\sqrt{1+\sqrt{a_n}}$, for all $n\in\N$. Show that the sequence $(a_n)_{n\in\N}$ converges. You are not asked to find its limit; however, can you find an equation that the limit satisfies, which will give the limit once solved?

\vspace{0.4in}
\end{enumerate}
\end{document}