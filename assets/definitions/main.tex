\documentclass{article}
\usepackage[utf8]{inputenc}
\usepackage[letterpaper, margin=1.25in]{geometry}
\usepackage{amsmath}
\usepackage{amsthm}
\usepackage{amsfonts}
\usepackage{amssymb}
\usepackage{enumerate}
\usepackage[utf8]{inputenc}
\usepackage[english]{babel}
\usepackage[document]{ragged2e}
\usepackage{physics}
\usepackage{graphics}
\usepackage{epsfig}
\usepackage{color}
\usepackage{verbatim}
\title{Definitions}
\author{Chin Chang}
\date{September 2016}

\begin{document}

\maketitle

\section{Lecture 1}
\begin{itemize}
    \item Natural numbers: $\mathbb{N} := \{ 1,2,3,\ldots$ \} \\
    \vspace{0.25cm} \textit{"Basic Definitions" , "Natural Numbers"}
    
    \item Integers: $ \mathbb{Z} := \{ \ldots, -3,-2,-1,0,1,2,3,\ldots \}$. \\
    \vspace{0.25cm} \textit{"Basic Definitions" , "Integers"}
    
    \item Rational numbers: $ \mathbb{Q} := \{ \frac{m}{n} : m \in \mathbb{Z} , n \in \mathbb{Q} \}$ \\
    \vspace{0.25cm} \textit{"Basic Definitions" , "Rational Numbers"}
    
    \item (Proposition) $\sqrt{2} \notin \mathbb{Q} $ \\
    \vspace{0.25cm} \textit{"sqrt(2)" , "Square root 2" , "Proposition"}
    
    \item (Definition) An \underline{operation} * on a set S $\neq \emptyset$ is a map $$*: S x S \rightarrow S$$ $$(a,b) \rightarrow a*b$$
    Note that $*$ is not multiplication; rather, it is a placeholder for some arbitrary operation (whether its addition or multiplication or exponent \\
    \vspace{0.25cm} \textit{"Basic Definitions" , "Maps" , "Definition"}
    
    \item (Definition) Let $\mathbb{F} \neq \emptyset$ be a set. Let $+,*$ be two operations on $\mathbb{F}$. We say that the triple $(\mathbb{F},+,*)$ is a field if $+$ and $*$ satisfy the following:
        \begin{enumerate}
            \item Axioms for $+$:
            \begin{enumerate}
                \item $a+b = b+a \hspace{0.1cm}, \forall a,b \in \mathbb{F}$ (commutativity)
                \item $a+(b+c) = (a+b)+c \hspace{0.1cm}, \forall a,b,c \in \mathbb{F}$ (associativity)
                \item There exists an element of $\mathbb{F}$, which we denote by 0, s.t. $$a+0 = a, \forall a \in \mathbb{F}$$ (existence of additive identity)
                \item $\forall a \in \mathbb{F}$, there exists a' $\in \mathbb{F}$ s.t. $$a+a' = 0$$ (existence of additive inverse)
            \end{enumerate}
            \item Axioms for $*$:
            \begin{enumerate}
                 \item $a*b = b*a \hspace{0.1cm}, \forall a,b \in \mathbb{F}$ (commutativity)
                \item $a*(b*c) = (a*b)*c \hspace{0.1cm}, \forall a,b,c \in \mathbb{F}$ (associativity)
                \item There exists an element of $\mathbb{F}$, which we denote by 1, s.t. $$a*1 = a, \forall a \in \mathbb{F}$$ (existence of multiplicative identity)
                \item $\forall a \neq 0 \in \mathbb{F}$, there exists a' $\in \mathbb{F}$ s.t. $$a*a' = 1$$ (existence of multiplicative inverse)
            \end{enumerate}
            \item Distributivity: $$ a*(b+c) = a*b + a*c \hspace{0.1cm} , \forall a,b,c \in \mathbb{F}$$
            
            
            
        \end{enumerate}
        
    \textit{"Field Axioms" , "Fields" , "Basic Definitions" , "Definition"}    
    
    \item (Definition) Let $(\mathbb{F},+,*)$ be a field. We say that it is \underline{ordered} if $\exists P \subseteq \mathbb{F}$ , s.t. 
    \begin{enumerate}
        \item $\forall a \in \mathbb{F}$ , exactly one of the following holds: 
        \begin{enumerate}
            \item $a \in P$
            \item $a=0$
            \item $-a \in P$
        \end{enumerate}
        \item $\forall a,b \in P, a+b \in P $and $a*b \in P$
    \end{enumerate}
    \textit{"Field" , "Ordered" , "Definition"}
    
    \item Given that the set P exists, we can refer to it as the \underline{set of positive elements} of $(\mathbb{F},+,*)$.\\
    \vspace{0.25cm} \textit{"Positive" , "Ordered"}
    
    \item (Definition) Let $(\mathbb{F},+,*)$ be an ordered field, with $P \subseteq \mathbb{F}$ as the chosen subset of positive elements. Then, we have an order $<$ on $\mathbb{F}$, defined as:
    $$\forall a,b \in \mathbb{F}, a < b \Leftrightarrow b+(-a) \in P$$
    \textit{"Inequality" , "Ordered", "Definition"}
    
    \item (Properties) Properties of ordered fields: \\
    Let $(\mathbb{F} , + , *)$ be an ordered field with order $<$. Then:
    \begin{enumerate}
        \item If $a,b \in \mathbb{F}$, then exactly one of the following holds:
        \begin{enumerate}
            \item $a<b$
            \item $a=b$
            \item $a>b$
        \end{enumerate}
        \item If $a>b$ and $b>c$, then $a>c$
        \item If $a>b$ and $c \in \mathbb{F}$, then $a+c > b+c$
        \item If $a>b$ and $c>0$, then $a*c > b*c$
        \item If $a>b$ and $c>d$, then $a+c > b+d$
        \item $1>0$ (the multiplicative identity is larger than the additive \\ identity).
        \item If $a>b \Rightarrow a-b > 0$ and $b>c \Rightarrow b-c>0$ then $(a-b) + (b-c)>0$, i.e. $a-c>0$
        
    \end{enumerate}
    \textit{"Properties" , "Ordered", "Inequality"}
    
    
\end{itemize}


\section{Lecture 2}
\begin{itemize}
    \item (Definition) Let $(\mathbb{F},+,*)$ be an ordered field, and $A \subseteq \mathbb{F}$. We say that A is \underline{bounded from above} if there exists $b \in \mathbb{F}$ s.t. $$ a \leq b, \forall a \in A$$
    \textit{"Bounded" , "Ordered" , "Definition"}
    
    \item (Definition) Suppose that $A  \neq \emptyset \subseteq \mathbb{F}$ is bounded from above. Suppose that $b \in \mathbb{F}$ is an upper bound of A. We say that b is a \underline{least upper bound of A} if $b \leq c$, for all c upper bounds of A. \\
    \vspace{.25cm} \textit{"Bounded" , "Ordered" , "Least Upper Bound" , "Definition"}
    
    \item (Definition) Let $(\mathbb{F},+,*,<)$ be an ordered field. We say that $(\mathbb{F},+,*,<)$ is \underline{complete} if every non-empty subset of $\mathbb{F}$ that is bounded from above has a least upper bound $( \in \mathbb{F})$. \\
    \vspace{.25cm} \textit{"Complete" , "Bounded" , "Ordered" , "Least Upper Bound" , "Definition"}
    
    \item (Properties) The ordered field $(\mathbb{Q},+,*,<)$ is \underline{not} complete. \\
    \vspace{.25cm} \textit{"Complete" , "Q" , "Properties"}
    
    \item (Theorem) (The Real Numbers) There exists an extension of $(\mathbb{Q},+,*,<)$ to a \underline{complete ordered field}
    $(\mathbb{R},+,*,<)$.
    In other words,
    \begin{itemize}
        \item $\mathbb{Q} \subseteq \mathbb{R}$
        \item The operations $+$ and $*$ on $\mathbb{R}$ when restricted on $\mathbb{Q}$, are the original operations $+$ and $*$ on $\mathbb{Q}$
        \item The order $<$ on $\mathbb{R}$, restricted on $\mathbb{Q}$, is the same as the order $<$ on $\mathbb{Q}$
        \item Every $A \subseteq \mathbb{R}, A \neq \emptyset$ that is bounded from above has a least upper bound $(\in \mathbb{R})$
    \end{itemize}
    There exists a \underline{unique} complete ordered field (up to isomorphism) \\
    \vspace{.25cm} \textit{"Theorem" , "Complete" , "Ordered" , "Bounds"}
    
    \item (Corollary) The extension of $(\mathbb{Q},+,*,<)$ to a complete ordered field is \underline{unique}. We call this unique extension the field of real numbers \\
    \vspace{.25cm} \textit{"Corollary" , "Complete" , "Ordered" , "Real Numbers"}
    
\end{itemize}
\section{Lecture 3}
\begin{itemize}
    \item (Definition) $\forall A \subseteq \mathbb{R}$ let supA (the supremum of A) := the least upper bound of A in $\mathbb{R}$. \\ 
    \vspace{.25cm} \textit{"Least Upper Bound" , "Supremum" , "supA" , "Bounded" , "Definition"}
    \item (Properties) For any r $>$ 0 in $\mathbb{R}$ and any $n \in \mathbb{N}$, there exists a unique $x \in \mathbb{R}$, $x >0$ with $x^n = r$. We denote by $r^{\frac{1}{n}}$ this unique positive real. \\
    \vspace{.25cm} \textit{"Properties"}
    
    \item The real numbers have the Archimedean property, which can be expressed in three ways:
    \begin{enumerate}
        \item $\mathbb{N}$ is not bounded from above in $\mathbb{R}$. This tells us that we can find as large natural numbers as we want.
        \item Let $a, \epsilon \in \mathbb{R},  \epsilon > 0$. Then, there exists $n \in \mathbb{N}$ with $ n* \epsilon >a$ This tells us that no matter how small $\epsilon$ was, we can always make it as large as we want by multiplying it with some number.
        \item Let $\epsilon >0$. Then, there exists $n \in \mathbb{N}$ such that $\frac{1}{n} < \epsilon$. This tells us that, no matter how small $\epsilon$ is, we can always divide 1 into so many equal line segments that each will be smaller than $\epsilon$.
        
    \end{enumerate}
    \textit{"Archimedean" , "Bounded" , "Natural Numbers"}
    
    \item (Proposition) (Existence of integer part of every real) Let $x \in \mathbb{R}$. There exists a unique integer $m \in \mathbb{Z}$, such that $m \leq x < m+1$. We say that this is \underline{the integer part of x}. \\
    \vspace{.25cm} \textit{"Proposition" , "Integer part" , "Floor"}
    
    \item (Proposition) (Denseness of $\mathbb{Q}$ in $\mathbb{R}$) For any $a,b \in \mathbb{R}$ with $a<b$, there exists $q \in \mathbb{Q}$ with $$a<q<b$$. \\
    \vspace{.25cm} \textit{"Proposition" , "Denseness" , "Q"}
    
    \item (Corollary) For any $a , b \in \mathbb{R}$ with $ a<b$, there exists \underline{infinitely many} rationals q with $$a<q<b$$ \\
    \vspace{.25cm} \textit{"Denseness" , "Q" , "Complete" , "Corollary"}
    
    \item (Definition) (Denseness of $\mathbb{R}$\textbackslash $\mathbb{Q}$ in $\mathbb{R}$)\\ We define the set of \underline{irrational numbers} to be $\mathbb{R}$\textbackslash $\mathbb{Q}$. \\
    \vspace{.25cm} \textit{"Irrational Numbers" , "Denseness" , "R mod Q" , "Definition"}
    
    \item (Proposition) For any $a,b \in \mathbb{R}$ with $a<b$, there exists $x \in \mathbb{R}$\textbackslash $\mathbb{Q}$ with $a<x<b$ \\
    \vspace{.25cm} \textit{"Proposition" , "Denseness" , "R mod Q"}
    
    \item (Corollary) For any $a,b \in \mathbb{R}$ with $a<b$ there exists infinitely many irrationals x with $a<x<b$.\\
    \vspace{.25cm} \textit{"Irrationals" , "Denseness" , "Corollary"}
    
    
\end{itemize}
\section{Lecture 4}
\begin{itemize}
    \item For any $a \in \mathbb{R}$, we define its absolute value as:
    $$ \abs{a} = 
    \begin{cases} 
        a , a \geq 0 \\
        -a , a <0 \\
        \end{cases} $$
 \\
\vspace{.25cm} \textit{"Absolute Value"}

    \item Bernoulli's Inequality: If $a \geq -1$, then $(1+a)^n \geq 1 +na, \forall n \in \mathbb{N}$ \\
    \vspace{.25cm} \textit{"Inequality" , "Bernoulli" , "Bernoulli's Inequality"}
    
    \item Binomial Expansion: If $a,b \in \mathbb{R}$ and $n \in \mathbb{N}$, then $(a+b)^n = \sum_{k=0}^na^kb^{n-k}$ \\
    \vspace{.25cm} \textit{"Inequality" , "Binomial" , "Binomial Expansion"}
    
    \item Cauchy-Schwarz Inequality: if $a_1,a_2,\ldots,a_n \in \mathbb{R}$ and $b_1,b_2,\ldots,b_n \in \mathbb{R}$, then:  $$(a_1b_1+a_2b_2+\ldots+a_nb_n)^2 \leq (a_1^2+a_2^2+\ldots+a_n^2)(b_1^2+b_2^2+\ldots+b_n^2)$$
    In other words, $$ \abs{\sum_{k=1}^na_kb_k} \leq \sqrt{\sum_{k=1}^na_k^2}\sqrt{\sum_{k=1}^nb_k^2}}$$
    \vspace{.25cm} \textit{"Cauchy" , "Schwarz" , "Cauchy Schwarz Inequality"}
    
    \item Arithmetic-Geometric-Harmonic Mean Inequality: If $x_1,x_2,\ldots,x_n > 0$, then $$\frac{n}{\frac{1}{x_1}+\frac{1}{x_2}+\ldots+\frac{1}{x_n}} \leq \sqrt[n]{x_1x_2\ldots x_n} \leq \frac{x_1+x_2+\ldots+x_n}{n}$$ With equality only if $x_1=x_2=\ldots=x_n$. \\
    \vspace{.25cm} \textit{"Arithmetic Mean" , "Geometric Mean" , "Harmonic Mean" , "Inequality"}
    
    \item A sequence is a map $a:\mathbb{N} \rightarrow \mathbb{R}$ We denote each a(n) by $a_n$, for simplicity. We also denote the sequence a by: $(a_n)_{n \in \mathbb{N}}$ or  $(a_n)_{n=1}^\infty$ or $(a_n)$ or $(a_1,a_2,a_3,\ldots)$. \\
    \vspace{.25cm} \textit{"Sequence"}
    
    \item A set of terms of the sequence $(a_n)_{n\in \mathbb{N}$ is the set $\big\{a_n:n\in \mathbb{N}\big\}$ \\
    \vspace{.25cm} \textit{"Sequence" , "Set of Terms"}
    
    \item If $(a_n)_{n=1}^\infty$ is a sequence and $m\in \mathbb{N}$, then the sequence $(a_m,a_{m+1},a_{m+2},\ldots)$ is called a final part of $(a_n)_{n=1}^\infty$. Note that $(a_m,a_{m+1},\ldots) = (a_{m+n-1})_{n\in \mathbb{N}$.\\
    \vspace{.25cm} \textit{"Final Part" , "Sequence"}
    
    \item Let $(a_n)_{n\in\mathbb{N}}$ be a sequence of real numbers, and $a\in \mathbb{R}$. We say that $(a_n)_{n\in\mathbb{N}}$ converges to a, and that a is the limit of $(a_n)_{n\in\mathbb{N}}$ and we write: "$a_n \rightarrow a$ as $n \rightarrow +\infty$", if: $$ \forall \epsilon > 0, \exists n_0 = n_0(\epsilon) \in \mathbb{N} : \forall n \geq n_0, \abs{a_n - a} < \epsilon$$
    
    We can rephrase this definition as: \\
    $(a_n)_{n\in\mathbb{N}}$ converges to a if, 
    for any neightborhood $(a-\epsilon , a+\epsilon)$ of a, there exists a final part of $(a_n)_{n \in \mathbb{N}}$ contained in $(a-\epsilon , a+\epsilon)$. \\
    \vspace{.25cm} \textit{"Convergence" , "Converge" , "Limit"}
    
    
\end{itemize}
\section{Lecture 5}
\begin{itemize}
    \item The sequence $(a_n)_{n\in\mathbb{N}}$ is:
    \begin{itemize}
        \item \underline{Bounded from above} if $\exists b \in \mathbb{R}$ s.t. $a_n \leq b \forall n \in\mathbb{N}$.
        \item \underline{Bounded from below} if $\exists c \in\mathbb{R}$ s.t. $a_n \geq c, \forall n \in\mathbb{N}$
        \item \underline{Bounded} if $\exists b,c \in\mathbb{R}$ s.t. $c \leq a_n \leq b , \forall n \in \mathbb{N}$
    \end{itemize}
    \vspace{.25cm} \textit{"Bounded" , "Bounded Sequence"}
    
\end{itemize}
\section{Lecture 6}
\section{Lecture 7}
\begin{itemize}
    \item $$e:= \lim_{n\to\infty}(1+\frac{1}{n})^n$$
    \textit{"Exponential"}
    
    \item We say that $a_n \rightarrow + \infty \text{ as } n \rightarrow +\infty$ if: $$\forall M >0, \exists n_0 = n_0(M) \in \mathbb{N} \text{ s.t.: } a_n >M, \forall n \geq n_0$$ This is equivalent to saying: $$a_n \rightarrow +\infty \text{ as } n \rightarrow +\infty \text{ if } \forall M >0, \text{there exists a whole final part of  } (a_n)_{n\in\mathbb{N}} \in (M,+\infty)$$ \\
    \textit{"Limit of Sequence" , "Limit"}
    
    \item The Ratio Test: Let $(a_n)_{n\in\mathbb{N}}$ be a sequence, with $a_n \neq 0,  \forall n \in \mathbb{N}$. Consider $b_n = \frac{\abs{a_{n+1}}}{\abs{a_n}}$ \\ 
    \begin{itemize}
        \item If $\lim_{n\rightarrow +\infty} b_n = p < 1$, then $a_n \rightarrow 0$
        \item If $\lim_{n\rightarrow +\infty} b_n = p > 1$, then $\abs{a_n} \rightarrow + \infty$
        \item If $\lim_{n\rightarrow +\infty} b_n = 1$, then the test is inconclusive
        
    \end{itemize}
    \textit{"Ratio Test"}
    
    \item The Root Test: Let $(a_n)_{n\in\mathbb{N}}$ be a sequence, with $a_n \neq 0,  \forall n \in \mathbb{N}$. Consider $b_n = \sqrt[n]{\abs{a_n}} 
    \begin{itemize}
        \item If $\lim_{n\rightarrow +\infty} b_n = p < 1$, then $a_n \rightarrow 0$
        \item If $\lim_{n\rightarrow +\infty} b_n = p > 1$, then $\abs{a_n} \rightarrow + \infty$
        \item If $\lim_{n\rightarrow +\infty} b_n = 1$, then the test is inconclusive
        
    \end{itemize} \\
    
    \textit{"Root Test"}
    
\end{itemize}
\section{Lecture 8}
\begin{itemize}
    \item Let $(a_n)_{n\in\mathbb{N}$ be a sequence of real numbers. A sequence $(b_n)_{n\in\mathbb{N}$ is called a subsequence of $(a_n)_{n\in\mathbb{N}$ if there exists $k_1 < k_2 < \ldots < k_n < k_{n+1} < \ldots \in \mathbb{N}$ s.t. $b_n=a_{k_n} , \forall n\in\mathbb{N}$\\
    \vspace{.25cm} \textit{"Subsequence"}
    
    \item (Bolzano-Weierstrass Theorem): Every Bounded sequence in $\mathbb{R}$ has a convergent subsequence \\
    \vspace{.25cm} \textit{"Bolzano Weierstrass"}
\end{itemize}
\section{Lecture 9}
\begin{itemize}
    \item A sequence $(a_n)_{n\in\mathbb{N}$ is a Cauchy sequence if: $\forall \epsilon > 0,  \exists n_0 \in \mathbb{N}$ s.t.: $\forall n \geq n_0, \abs{a_n - a_m} < \epsilon$
    \vspace{.25cm} \textit{"Cauchy Sequence" , "Cauchy"}
    
    \item Let $(a_n)_{n\in\mathbb{N}$ be a sequence of real numbers:
    
\end{itemize}
\end{document}
